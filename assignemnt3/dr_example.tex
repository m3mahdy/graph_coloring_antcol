\documentclass{article}
\usepackage[utf8]{inputenc}
\usepackage{amsmath}
\usepackage{amssymb}
\usepackage{graphicx}
\usepackage{booktabs} % For professional looking tables
\usepackage{geometry}
\geometry{a4paper, margin=1in}

\begin{document}

% --- Title Page Content (Simulated) ---
\begin{titlepage}
    \centering
    \includegraphics[width=2cm]{KSU_Logo.png} % Placeholder for University Logo
    \vspace{0.5cm}
    {\large King Saud University \\ College of Computer and Information Sciences \\ Computer Science Department}
    \vspace{1cm}
    
    \hrule
    \vspace{0.5cm}
    {\Huge\bfseries Assignment 3}
    \vspace{0.5cm}
    {\Large Population based metaheuristic to solve \\ Capacitated Vehicle Routing Problem (CVRP)}
    \vspace{0.5cm}
    \hrule
    \vspace{2cm}
    
    \begin{flushleft}
        \textbf{Course Title:} Selected topics in Artificial Intelligence \\
        \textbf{Course Code:} CS569
    \end{flushleft}
    
    \vfill
    \begin{center}
        \text{1/14} % Fictional page numbering as the source starts with 1/14 but has content on page 1.
    \end{center}
\end{titlepage}
\setcounter{page}{1}
% \maketitle
\thispagestyle{empty}
% --- End Title Page Content ---

\pagebreak

\section{Problem Statement}

The \textbf{Vehicle Routing Problem (VRP)} is one of the most important and widely studied problems due to its application in delivery and transport logistics[17]. VRP solutions have specific objectives and limitations in real applications, making VRP have categories or variants[18]. One of the most popular VRP variants is \textbf{Capacitated Vehicle Routing Problem (CVRP)}[19]. This problem is considered a fundamental problem in combinatorial optimization[20]. It is known to be an \textbf{NP-hard} problem and a generalization of the Traveling Salesman Problem[21].

CVRP can be described as follows: there is a set of geographically dispersed customers, where each customer has a certain demand[22]. There is also one centralized depot and the travel costs between all locations are given[23]. To serve customers' demands, there is a fleet of identical vehicles whose base is the depot, and each vehicle has a certain capacity[24]. The main goal of this problem is to find the \textbf{minimum cost routes} for the capacitated vehicles to serve customers' demands[25].

The constraints are as follows:
\begin{enumerate}
    \item Each customer is visited exactly once and served by exactly one vehicle[26].
    \item Each vehicle starts at and returns to the depot after serving a subset of customers[27].
    \item Each customer's demand is satisfied and the total load on any route does not exceed the vehicle's capacity[28].
\end{enumerate}

\section{Methodology (P-metaheuristic)}

This section shows the details of the used \textbf{Population-Based metaheuristics (P-metaheuristics)} to solve CVRP, which is the \textbf{Genetic Algorithm (GA)}[30, 31]. GA is developed by John Holland in Michigan, USA in 1975[32]. GA is an optimization algorithm inspired by the principles of natural evolution and genetics, and it is one of the most common evolutionary algorithms[32].

\subsection{Main Concepts in GA}
The main concepts in GA are:
\begin{itemize}
    \item \textbf{Chromosome/Genotype}: the representation of a potential solution. It is composed of genes that encode the parameters or features of the solution[34].
    \item \textbf{Population}: a collection of individuals, each representing a potential solution to the optimization problem[35].
    \item \textbf{Fitness Function}: an objective function that evaluates how well a solution performs with respect to the optimization criteria[36].
    \item \textbf{Crossover (Recombination)}: the process of exchanging genetic material between two individuals to create new offspring[42]. The main role of crossover is to inherit some characteristics of the two parents to generate the offspring[43]. The crossover rate is commonly in the interval $[0.45, 0.95]$[44].
    \item \textbf{Mutation}: the introduction of small random changes to the genetic material of an individual to maintain diversity in the population[45]. It represents small changes of the selected individual[46]. The Mutation Rate is commonly in the interval $[0.001, 0.01]$[46].
    \item \textbf{Selection}: the process of choosing individuals from the current population to become parents for the next generation[47]. Individuals with a higher fitness value have more chance of being selected[48]. A replacement strategy is needed to decide which individuals will survive to the next generation[49].
\end{itemize}

As a heuristic algorithm, GA needs a termination criterion[50, 51]. The used GA stopping criterion is stopping after a maximum number of iterations (\textbf{4000})[52].

\pagebreak

\subsection*{Algorithm 1 Genetic Algorithm (GA) algorithm}
\begin{enumerate}
    \item $\text{population\_size} \leftarrow \text{desire population size}$ [53]
    \item $\text{GenerationSize} \leftarrow \text{desire generation size}$ [53]
    \item $\text{population} \leftarrow \{\}$ [53]
    \item $\text{for population\_size time do}$ [53]
    \item $\text{for} (i=0; i < \text{population\_size}/2; i++)$ \text{do} [53]
    \item $\text{population} \leftarrow \text{population} \cup \{\text{Nearest Neighbor individual}\}$ [53]
    \item $\text{for} (i=\text{population\_size}/2; i < \text{population\_size}; i++)$ \text{do} [58]
    \item $\text{population} \leftarrow \text{population} \cup \{\text{Random individual}\}$ [60, 61]
    \item $\text{Best} \leftarrow \text{None}$ [62]
    \item $\text{Repeat}$ [63]
    \item $\text{for each individual } P \in \text{population do}$ [65]
    \item $\text{Calculate\_fitness} (P)$ [69]
    \item $\text{if } \text{Best} = \text{None} \text{ or } \text{Calculate\_fitness} (P) > \text{Calculate\_fitness} (\text{Best}) \text{ then}$ [70]
    \item $\text{Best} \leftarrow P$ [71]
    \item $\text{offspring} \leftarrow \{\}$ [73]
    \item $\text{for population\_size}/2 \text{ do}$ [75]
    \item $\text{Parents} \leftarrow \text{Roulette\_wheel\_selection}(\text{population})$ [77, 78]
    \item $\text{children} \leftarrow \text{crossover}(\text{copy}(\text{Parents}))$ [80, 81]
    \item $\text{offsprings} \leftarrow \text{offsprings} \cup \text{mutation} (\text{children})$ [83, 84]
    \item $\text{population} \leftarrow \text{half of population} \cup \text{half of offsprings}$ [85]
    \item $\text{Until reach GenerationSize}$ [86]
    \item $\text{return Best}$ [87]
\end{enumerate}

\subsection{Solution Representation}
One solution representation method for CVRP is \textbf{permutation-based}[93]. A permutation of the depot (0) and $N$ customers is used to represent a solution, where 0 can appear more than once for a new route, and each customer can only occur once[94]. A one-to-one representation-solution mapping function is used[97].

\paragraph{Example of a Solution (Instance A-n32-k5):}
The solution is determined to have 5 routes for 31 customers (0 represents the depot)[95, 96].
\begin{verbatim}
[
 [0, 27, 24, 14, 26, 30, 16, 0],
 [0, 21, 31, 19, 17, 13, 7, 0],
 [0, 8, 18, 22, 9, 15, 10, 25, 5, 29, 20, 0],
 [0, 2, 3, 23, 28, 4, 11, 6, 0],
 [0, 12, 1, 0]
]
\end{verbatim}

\subsection{Fitness Function}
The used fitness function is the \textbf{inverse of the objective function}, which computes the total route distance for all routes in the solution[113].

The \textbf{objective function} is the total distance for all routes in the solution[116]. The total distance for each route is calculated by: (departing distance from depot to first node) + (distances between all following nodes) + (returning distance from last node to the depot)[115]. The goal is minimization of this objective function[117].

The objective function $F$ is given by:
$$F=\sum_{r=1}^{v}\left(d_{\pi(0),\pi(1)}+\sum_{i=1}^{n-1}d_{\pi(i),\pi(i+1)}+d_{\pi(n),\pi(0)}\right) \text{ [118]}$$
where $v$ is the number of vehicles, $n$ is the number of customers, $\pi$ represents a permutation encoding of a route $r$, and $\pi(0)$ represents the depot[118, 119].

The \textbf{fitness function} is:
$$\frac{1}{F} \text{ [120]}$$

\subsection{Constraint Handling}
The most important constraint is the \textbf{vehicle capacity}: the total load of a vehicle (summation of route customers' demands) must not exceed the vehicle capacity[122].

The GA implements a \textbf{preserving strategy} for constraint handling, which ensures the generation of feasible solutions through specific representation and operators[123].
\begin{itemize}
    \item The initial population consists of a set of initial feasible solutions where each vehicle load satisfies the capacity[124].
    \item After crossover, the produced nodes in the permutation are arranged in routes according to vehicle capacity[125].
    \item For mutation, only feasible solutions will be generated because the Swap (intra-route) operator is used[126].
\end{itemize}

\pagebreak

\subsection{Initial Population}
The initial population is a collection of potential solutions (individuals/chromosomes)[128].
\begin{itemize}
    \item \textbf{50\%} of the population size is generated using the \textbf{Nearest Neighbor (NN)} algorithm[129].
    \item The other \textbf{50\%} is generated \textbf{randomly} to improve the diversification of the algorithm[136].
\end{itemize}
In the NN algorithm, each route is initialized with a randomly chosen node and continues until all vehicle capacities are consumed or all demands are satisfied[137].

\subsection*{Algorithm 2 Nearest Neighbor (NN) algorithm}
\begin{enumerate}
    \item Input: Demand list $D$, Number of vehicles $k$, Vehicle Capacity ($C$), List of Unvisited Nodes ($N$) [141, 142]
    \item $\text{Initialize an empty Routes list}$ [142]
    \item $v=0$ [142]
    \item $\text{While } v<k \text{ Do}$ [142]
    \item $\quad \text{route} = []$ [142]
    \item $\quad \text{capacity} = 0$ [142]
    \item $\quad \text{idx} = \text{random unvisited node}$ [142]
    \item $\quad \text{Append idx to route}$ [142]
    \item $\quad \text{capacity} = \text{capacity} + D[\text{idx}]$ [142]
    \item $\quad \text{While there is unvisited node and capacity} < C \text{ Do}$ [142]
    \item $\quad \quad j = \text{Nearest Unvisited Node}$ [142]
    \item $\quad \quad \text{If } (\text{capacity} + D[j] < C) \text{ Then}$ [142]
    \item $\quad \quad \quad \text{Append } j \text{ to route}$ [142]
    \item $\quad \quad \quad \text{capacity} = \text{capacity} + D[j]$ [142]
    \item $\quad \quad \text{Else}$ [142]
    \item $\quad \quad \quad \text{Break}$ [142]
    \item $\quad \quad \text{End while}$ [142]
    \item $\quad v=v+1$ [142]
    \item $\quad \text{Append route to Routes list}$ [142]
    \item $\text{End while}$ [142]
    \item $\text{Return Routes list}$ [142]
\end{enumerate}
The output is a feasible solution where each vehicle's total load satisfies the capacity[146].

\subsection{Parents Selection}
The \textbf{Roulette Wheel Selection} method is used[152].
\begin{enumerate}
    \item Calculate the fitness ($f_i$) of each individual[153].
    \item Normalize the fitness values to convert them into probabilities ($p_i$)[154, 156]:
    $$p_{i}=\frac{f_{i}}{\sum_{j=1}^{n}f_{j}} \text{ [157]}$$
    \item Generate a random number between 0 and 1[158].
    \item Select the individual whose probability slice contains the randomly generated number (larger fitness $\implies$ larger slice $\implies$ higher chance of selection)[160, 161].
    \item Repeat the process to select the desired number of individuals; the same individual can be selected more than once[162, 163].
\end{enumerate}

\pagebreak

\subsection{Crossover Operator}
The crossover operator used for permutation representation is \textbf{Order Crossover}[165, 166].

\paragraph{Order Crossover Steps:}
\begin{enumerate}
    \item Two crossover points are randomly selected from \textbf{parent 1}[167].
    \item The part between the two points is copied from parent 1 to the \textbf{offspring 1}[168].
    \item From \textbf{parent 2}, start at the first node and pick the elements that are not already selected (copied from parent 1) to fill the remaining parts of offspring 1[169, 170].
    \item Two crossover points are randomly selected from \textbf{parent 2}[175].
    \item The part between the two points will be copied from \textbf{parent 1} to the \textbf{offspring 2}[176].
    \item From \textbf{parent 1}, start at the first node and pick the elements that are not already selected (copied from parent 1) to fill the remaining parts of offspring 2[179, 180]. (Note: Step 6 repeats parent 1/parent 1, which appears to be a transcription error in the original text, likely meant to be "From parent 2, start at the first node and pick elements..." for a standard Order Crossover approach, but the source explicitly states parent 1/parent 1 for offspring 2's filling sequence [179, 180]).
\end{enumerate}

\subsection{Mutation}
The mutation operator used is \textbf{Swap (intra-route)}: swap two randomly chosen nodes from one randomly chosen route[251].

\paragraph{Example (Selected route is 3, selected nodes are 2 and 4):}
\begin{verbatim}
Before mutation:
[[4, 11, 8, 18, 22, 9, 15, 10], [13, 7, 16, 26, 30, 12], 
[28, 23, 3, 2, 6, 14, 24], [29, 5, 20, 27, 1, 21, 31], [17, 19, 25]]

After mutation (nodes 2 and 4 swapped in route 3):
[[4, 11, 8, 18, 22, 9, 15, 10], [13, 7, 16, 26, 30, 12], 
[28, 23, 3, 6, 14, 2, 24], [29, 5, 20, 27, 1, 21, 31], [17, 19, 25]] 
\end{verbatim}

\subsection{Replacement}
The replacement strategy used is replacing approximately \textbf{50\% of the old population} with the new offsprings[262, 267].

\subsection{Diversification and Intensification strategies}
\begin{itemize}
    \item \textbf{Diversification} (Exploration): Starting with a diverse initial population helps with exploration[273]. Mutation operators generally generate worse offspring, which increases diversification[274].
    \item \textbf{Intensification} (Exploitation): Crossover operators mostly produce children that are at least better than their worse parent, increasing the intensification of the algorithm[275, 276].
\end{itemize}

\section{Motivation}
CVRP is an NP-hard problem, so exact methods can only solve small instances in a reasonable time[278]. Metaheuristic methods are required for larger instances and real-life transportation problems to find optimal or near-optimal routes efficiently[279, 280]. GA is one of the commonly used P-metaheuristics that is attractive for solving optimization problems because it can obtain good solutions in reasonable time[281]. The work also aims to compare the results of GA and Tabu Search (TS)[282].

\section{Experimental Setup}

\subsection{Implementation Environment}
The experiment was implemented using the \textbf{Python} programming language and run on the \textbf{Google Colab} platform, with a 1 core 2.25 GHz CPU (AMD EPYC 7B12) and 13 GB RAM[293].

\pagebreak

\subsection{Parameter Tuning}
The \textbf{offline parameter tuning} strategy was used, where parameters were tuned one at a time to empirically determine their optimal value[295].

\paragraph{Table 1: Values of parameters tuning. [297]}
\begin{table}[h]
    \centering
    \begin{tabular}{lcc}
        \toprule
        \textbf{Parameters (Factors)} & \textbf{Levels} & \textbf{Optimal value} \\
        \midrule
        Generation & 500, 1000, 4000, 5000 & 4000 \\
        Population Size & 50, 70, 100 & 100 \\
        Crossover Rate & 0.8, 0.85, 0.9 & 0.9 \\
        Mutation Rate & 0.08, 0.09, 0.1 & 0.1 \\
        \bottomrule
    \end{tabular}
\end{table}

\subsection{Datasets}
Three benchmark instances from the \textbf{CVRPLIB} public library were used: \textbf{A-n32-k5}, \textbf{F-n135-k7}, and \textbf{M-n200-k17}[301].

\paragraph{Table 2: The used CVRP benchmark instances. [304]}
\begin{table}[h]
    \centering
    \begin{tabular}{lccc}
        \toprule
        \textbf{Instance} & \textbf{Nodes number} & \textbf{Vehicles number} & \textbf{Capacity} \\
        \midrule
        A-n32-k5 & 31 & 5 & 100 \\
        F-n135-k7 & 134 & 7 & 2210 \\
        M-n200-k17 & 199 & 17 & 200 \\
        \bottomrule
    \end{tabular}
\end{table}

\begin{figure}[h]
    \centering
        \caption{A-n32-k5 Instance.}
\end{figure}

\begin{figure}[h]
    \centering
        \caption{F-n135-k7 Instance.}
\end{figure}

\begin{figure}[h]
    \centering
        \caption{M-n200-k17 Instance.}
\end{figure}

\pagebreak

\section{Results and Discussion}

\subsection{Performance Measurement}
\begin{itemize}
    \item \textbf{Solution Quality}: Measured by the percent deviation of the obtained solution from the \textbf{Best-Known Solutions (BKS)} from CVRPLIB[367, 368].
    \item \textbf{Computational Effort}: Measured by the clock time (without I/O and pre/post-processing) [370] and the number of objective function evaluations (independent of the computer system)[371, 372].
    \item \textbf{Robustness}: Measured by performing 7 different runs on the same instance and taking the average of the obtained results, due to the randomness in the algorithm[372, 373].
\end{itemize}
The GA results are compared with the results achieved in Assignment 2 using \textbf{Tabu Search (TS)}[374].

\subsection{Solution Quality Comparison (GA vs. TS)}
The difference between the best, worst, and average results for GA is considered reasonable, similar to TS[377]. The GA efficiency decreases as the instance size increases, which is expected due to the problem's complexity increasing with size[382].

\begin{itemize}
    \item \textbf{A-n32-k5}: Best GA objective is 849.30 (8.3\% worse than BKS of 784.00)[379, 406].
    \item \textbf{F-n135-k7}: Best GA objective is 1301.39 (12\% worse than BKS of 1162.00)[380, 406].
    \item \textbf{M-n200-k17}: Best GA objective is 1542.37 (21\% worse than BKS of 1275.00)[381, 406].
\end{itemize}

TS outperforms GA in terms of solution quality for all instances, based on the best obtained results and their averages[384]. However, GA achieved a better worst obtained result for the F-n135-k7 instance[385].

\paragraph{Table 3: Tabu search and Genetic Algorithm solutions quality. [395]}
\begin{table}[h]
    \centering
    \scriptsize
    \begin{tabular}{ccccccc}
        \toprule
        & \multicolumn{3}{c}{\textbf{Tabu search}} & \multicolumn{3}{c}{\textbf{Genetic Algorithm}} \\
        \cmidrule(lr){2-4} \cmidrule(lr){5-7}
        \textbf{Run \#} & \textbf{A-n32-k5} & \textbf{F-n135-k7} & \textbf{M-n200-k17} & \textbf{A-n32-k5} & \textbf{F-n135-k7} & \textbf{M-n200-k17} \\
        \midrule
        1 & 816.68 & 1296.12 & 1465.56 & 849.30 & 1307.10 & 1559.47 \\
        2 & 842.81 & 1282.99 & 1501.58 & 854.72 & 1310.20 & 1571.91 \\
        3 & 818.57 & 1353.35 & 1564.06 & 854.18 & 1315.90 & 1547.80 \\
        4 & 849.44 & 1309.37 & 1553.67 & 854.72 & 1315.96 & 1571.56 \\
        5 & \textbf{807.70} & 1300.89 & 1478.67 & 856.25 & 1301.39 & 1559.19 \\
        6 & 839.93 & \textbf{1281.21} & 1524.78 & 854.74 & 1303.85 & \textbf{1542.37} \\
        7 & 816.68 & 1299.16 & 1494.28 & 856.25 & 1303.23 & 1561.01 \\
        \midrule
        \textbf{Best} & \textbf{807.70} & \textbf{1281.21} & \textbf{1465.56} & 849.30 & 1301.39 & 1542.37 \\
        \textbf{Worst} & 849.44 & 1353.35 & 1564.06 & 856.25 & \textbf{1315.96} & 1571.91 \\
        \textbf{Avg.} & \textbf{827.40} & \textbf{1303.30} & \textbf{1511.80} & 854.31 & 1308.23 & 1559.04 \\
        \bottomrule
    \end{tabular}
    \caption{Tabu search and Genetic Algorithm solutions quality. (*Bold indicate best result)}
\end{table}

\pagebreak

\paragraph{Table 4: Comparison between Tabu search solutions, Genetic Algorithm solutions and best-known solutions for CVRP. [405]}
\begin{table}[h]
    \centering
    \scriptsize
    \begin{tabular}{lccccccccc}
        \toprule
        \textbf{Instance} & \textbf{BKS} & \multicolumn{2}{c}{\textbf{Tabu search}} & \multicolumn{2}{c}{\textbf{Best Diff (\%)}} & \multicolumn{2}{c}{\textbf{Genetic Algorithm}} & \multicolumn{2}{c}{\textbf{Avg. Diff (\%)}} \\
        \cmidrule(lr){3-4} \cmidrule(lr){5-6} \cmidrule(lr){7-8} \cmidrule(lr){9-10}
        & & \textbf{Best} & \textbf{Avg.} & \textbf{Best} & \textbf{Avg.} & \textbf{Best} & \textbf{Avg.} & \textbf{Best} & \textbf{Avg.} \\
        \midrule
        A-n32-k5 & 784.00 & \textbf{807.70} & \textbf{827.40} & \textbf{3.0} & \textbf{5.5} & 849.30 & 856.04 & 8.3 & 9.2 \\
        F-n135-k7 & 1162.00 & \textbf{1281.21} & \textbf{1303.30} & \textbf{10.3} & \textbf{12.2} & 1301.39 & 1308.23 & 12.0 & 12.6 \\
        M-n200-k17 & 1275.00 & \textbf{1465.56} & \textbf{1511.80} & \textbf{14.9} & \textbf{18.6} & 1542.37 & 1559.04 & 21.0 & 22.3 \\
        \bottomrule
    \end{tabular}
    \caption{Comparison between Tabu search solutions, Genetic Algorithm solutions and best-known solutions for CVRP. (*Bold indicate best result)}
\end{table}

\begin{figure}[h]
    \centering
        \caption{Best solutions of TS and GA compared to BKS.}
\end{figure}

\pagebreak

\subsection{Computational Time and Objective Function Evaluation}
\begin{itemize}
    \item \textbf{Computational Time}: The average computational time of GA is \textbf{lower} compared to TS, thus GA outperforms TS in this criterion[392]. The long computational time for F-n135-k7 is due to the small number of vehicles compared to the number of nodes, resulting in longer routes[391].
    \item \textbf{Objective Function Evaluation}: GA shows a constant number of evaluations (400,000 for each run) because it evaluates the whole population at each generation for selection and finding the best individual[393, 435]. GA is considered more stable in this metric than TS, and GA is better for all instances except the best case of A-n32-k5[394].
\end{itemize}

\paragraph{Table 5: Computational time in seconds for Tabu search and genetic algorithm. [425]}
\begin{table}[h]
    \centering
    \scriptsize
    \begin{tabular}{ccccccc}
        \toprule
        & \multicolumn{3}{c}{\textbf{Tabu Search}} & \multicolumn{3}{c}{\textbf{Genetic Algorithm}} \\
        \cmidrule(lr){2-4} \cmidrule(lr){5-7}
        \textbf{Run \#} & \textbf{A-n32-k5} & \textbf{F-n135-k7} & \textbf{M-n200-k17} & \textbf{A-n32-k5} & \textbf{F-n135-k7} & \textbf{M-n200-k17} \\
        \midrule
        1 & 0:03:08 & 1:27:50 & 0:28:48 & 0:00:51 & 0:29:52 & 0:10:08 \\
        2 & 0:02:55 & 2:01:18 & 0:21:39 & 0:00:43 & 0:28:27 & 0:10:23 \\
        3 & 0:03:56 & 1:52:07 & 0:23:32 & \textbf{0:00:40} & 0:27:35 & 0:10:01 \\
        4 & 0:02:57 & 1:03:59 & 0:23:17 & 0:00:41 & \textbf{0:27:21} & 0:10:15 \\
        5 & 0:03:10 & 1:41:22 & 0:19:14 & 0:00:41 & 0:27:45 & 0:09:58 \\
        6 & 0:02:58 & 1:53:25 & 0:34:50 & 0:00:42 & 0:28:08 & \textbf{0:10:59} \\
        7 & 0:03:02 & 1:27:12 & 0:18:46 & 0:00:42 & 0:28:22 & \textbf{0:09:45} \\
        \midrule
        \textbf{Best} & \textbf{0:02:55} & \textbf{1:03:59} & \textbf{0:18:46} & \textbf{0:00:40} & \textbf{0:27:21} & \textbf{0:09:45} \\
        \textbf{Worst} & 0:03:56 & 2:01:18 & 0:34:50 & 0:00:51 & 0:29:52 & 0:10:59 \\
        \textbf{Avg.} & 0:03:10 & 1:38:10 & 0:24:18 & \textbf{0:00:43} & \textbf{0:28:13} & \textbf{0:10:13} \\
        \bottomrule
    \end{tabular}
    \caption{Computational time in seconds for Tabu search and genetic algorithm. (*Bold indicate best result)}
\end{table}

\paragraph{Table 6: Number of objective function evaluation for Tabu search and genetic algorithm. [431]}
\begin{table}[h]
    \centering
    \scriptsize
    \begin{tabular}{ccccccc}
        \toprule
        & \multicolumn{3}{c}{\textbf{Tabu search}} & \multicolumn{3}{c}{\textbf{Genetic Algorithm}} \\
        \cmidrule(lr){2-4} \cmidrule(lr){5-7}
        \textbf{Run \#} & \textbf{A-n32-k5} & \textbf{F-n135-k7} & \textbf{M-n200-k17} & \textbf{A-n32-k5} & \textbf{F-n135-k7} & \textbf{M-n200-k17} \\
        \midrule
        1 & \textbf{379357} & 1401036 & 664468 & 400000 & 400000 & 400000 \\
        2 & 388088 & 2083798 & 738333 & 400000 & 400000 & 400000 \\
        3 & 485945 & 1421682 & 720146 & 400000 & 400000 & 400000 \\
        4 & 389232 & \textbf{1117927} & 640540 & 400000 & 400000 & 400000 \\
        5 & 396196 & 1481537 & 1155677 & 400000 & 400000 & 400000 \\
        6 & 401504 & 1901465 & \textbf{625546} & 400000 & 400000 & 400000 \\
        7 & 383833 & 1214799 & 988878 & 400000 & 400000 & 400000 \\
        \midrule
        \textbf{Best} & \textbf{379357.00} & \textbf{1117927.00} & \textbf{625546.00} & 400000.00 & 400000.00 & 400000.00 \\
        \textbf{Worst} & 485945.00 & 2083798.00 & 1155677.00 & \textbf{400000.00} & \textbf{400000.00} & \textbf{400000.00} \\
        \textbf{Avg.} & 403450.71 & 1517463.43 & 790512.57 & \textbf{400000.00} & \textbf{400000.00} & \textbf{400000.00} \\
        \bottomrule
    \end{tabular}
    \caption{Number of objective function evaluation for Tabu search and genetic algorithm. (*Bold indicate best result)}
\end{table}

\pagebreak

\subsection{Best Obtained Solutions and Routes}
The following presents the best obtained solution for each of the three instances, including their plotted routes.

\begin{figure}[h]
    \centering
        \caption{Best solution for instance A-n32-k5.}
\end{figure}
\textbf{Routes of best solution for A-n32-k5:} [439]
\begin{verbatim}
[[21, 31, 19, 17, 13, 7],
 [27, 24, 14, 26, 30, 16],
 [8, 18, 22, 9, 15, 10, 25, 5, 29, 20],
 [6, 3, 2, 23, 28, 4, 11],
 [12, 1]]
\end{verbatim}

\begin{figure}[h]
    \centering
        \caption{Best solution for instance F-n135-k7.}
\end{figure}
\textbf{Routes of best solution for F-n135-k7:} [447]
\begin{verbatim}
[[26, 25, 21, 91, 22, 24, 23, 72, 73, 74, 76, 134, 32, 48, 1, 75, 47, 77, 64, 
  78, 63, 79, 34, 49, 62, 52, 51, 50, 53, 102, 103, 56, 57, 105, 104, 101, 100, 
  98],
 [6, 5, 4, 9, 7, 11, 12, 10, 2, 42, 41, 40, 44, 43, 39, 38, 96, 97, 99, 3, 18, 
  118, 46],
 [80, 33, 68, 69, 70, 67, 133, 66, 71, 17, 14, 88, 92, 28, 30, 31, 59, 60, 61, 
  45, 94, 93, 29],
 [83, 85, 84, 86, 87, 89, 90, 16, 13, 15, 54, 55, 58, 27],
 [120, 109, 108, 107, 106, 114, 115],
 [111, 125, 112, 126, 127, 128, 129, 113, 119, 117, 131, 116, 132],
 [82, 20, 19, 130, 65, 124, 123, 122, 110, 121, 81]]
\end{verbatim}

\begin{figure}[h]
    \centering
        \caption{Best solution for instance M-n200-k17.}
\end{figure}
\textbf{Routes of best solution for M-n200-k17:} [466]
\begin{verbatim}
[[129, 79, 185, 33, 81, 120, 9, 161, 103, 188],
 [116, 196, 76, 184, 77, 3, 158, 68, 150, 80, 177, 109],
 [82, 48, 124, 47, 168, 123, 19, 107, 175, 11, 126, 63, 90],
 [171, 74, 73, 21, 198, 110],
 [92, 151, 59, 99, 104, 96, 94, 183, 6, 147, 89, 166],
 [172, 42, 142, 14, 119, 192, 44, 141, 191, 91, 193, 100, 37, 98],
 [176, 1, 132, 69, 162, 101, 70, 30, 122, 51, 20, 128, 160, 131, 170],
 [5, 118, 60, 83, 199, 114, 8, 174, 125, 45, 17],
 [164, 34, 78, 169, 121, 29, 24, 163, 134, 54, 130, 165, 55, 25, 32],
 [194, 106, 153, 7, 182, 148, 62, 159, 10, 189, 108],
 [186, 56, 139, 187, 39, 23, 67, 4, 155, 179],
 [85, 93, 61, 173, 84, 113, 16, 86, 140, 38, 43],
 [152, 58, 40, 180, 26, 149, 195, 12, 154, 138],
 [137, 13, 117, 95, 97, 87, 144, 57, 178, 2, 115, 145, 41, 15, 53],
 [65, 136, 35, 135, 71, 66, 181, 64, 49, 143, 36, 46, 18, 52],
 [190, 127, 31, 88, 167, 27, 146, 156, 112, 105],
 [102, 157, 50, 111, 28]]
\end{verbatim}

\section{References}
\begin{enumerate}
    \item P. Toth and D. Vigo, The Vehicle Routing Problem. Society for Industrial and Applied Mathematics, 2002. doi: 10.1137/1.9780898718515. [587]
    \item Z. Borcinova, 'Two models of the capacitated vehicle routing problem', Croat. Oper. Res. Rev., vol. 8, pp. 463-469, Dec. 2017, doi: 10.17535/crorr.2017.0029. [588, 589]
    \item G. Laporte, 'What you should know about the vehicle routing problem', Nav. Res. Logist. NRL, vol. 54, no. 8, pp. 811-819, 2007, doi: 10.1002/nav.20261. [590, 591]
    \item J. H. Holland, Adaptation in Natural and Artificial Systems: An Introductory Analysis with Applications to Biology, Control, and Artificial Intelligence. MIT Press, 1992. [592, 593]
    \item Talbi and E.-G. Talbi, Metaheuristics: From Design to Implementation, vol. 74. 2009. doi: 10.1002/9780470496916. [594]
    \item A. Scheibenpflug and S. Wagner, 'An Analysis of the Intensification and Diversification Behavior of Different Operators for Genetic Algorithms', in Computer Aided Systems Theory - EUROCAST 2013, R. Moreno-Díaz, F. Pichler, and A. Quesada-Arencibia, Eds., in Lecture Notes in Computer Science. Berlin, Heidelberg: Springer, 2013, pp. 364-371. doi: 10.1007/978-3-642-53856-8\_46. [595, 596]
    \item 'CVRPLIB All Instances'. Accessed: Nov. 14, 2023. [Online]. Available: http://vrp.galgos.inf.puc-rio.br/index.php/en/ [597]
\end{enumerate}

\end{document}